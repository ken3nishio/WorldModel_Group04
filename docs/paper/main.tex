\documentclass[twocolumn]{ujarticle}
% \usepackage{jsaiac}
\usepackage[dvipdfmx]{graphicx}
\usepackage{url}
\usepackage{amsmath}
\usepackage{bm}
% \usepackage{algorithm}
% \usepackage{algorithmic}

\title{大規模動画生成モデルにおける適応的ガイダンスと\\
時間的忘却を用いた動的状態遷移の制御}

\author{
著者名 \\ 所属
}

\begin{abstract}
(前略)
\end{abstract}

\begin{document}
\maketitle

\section{はじめに}
(前略)

\section{関連研究}
(前略)

\section{問題設定と課題分析}
本節では、提案手法の導入に先立ち、最新の動画生成モデルが抱える構造的な課題である "Static Death"(静的死)を定義し、既存のアプローチではなぜ解決困難であるかを分析する。

\subsection{動画生成における「静的死 (Static Death)」}
拡散モデル(Diffusion Models)を用いた動画生成において、時間的一貫性(Temporal Consistency)は品質の根幹をなす要素である。HunyuanVideo等のモデルは、膨大な自然動画データセットから学習することで、隣接フレーム間の変化を滑らかに保つ強力な事前分布(Prior)を獲得している。
これを数理的に解釈すると、生成される動画の状態ベクトル系列 $\bm{x} = \{x_1, \dots, x_T\}$ に対し、以下のようなエネルギー関数 $E(\bm{x})$ を最小化するようなバイアスがかかっていると言える。
\begin{equation}
E(\bm{x}) \approx \sum_{t=1}^{T-1} ||x_{t+1} - x_t||^2
\end{equation}
この項は、フレーム間の急激な変化(高い運動エネルギー)にペナルティを与える「慣性項」として機能する。
一般的なシーン(歩行や風景)ではこの慣性が極めて有効であるが、「人物が消失する(Disappearance)」や「静止状態から突然高く跳躍する(Backflip)」といったタスクにおいては、この慣性が障壁となる。モデルは $x_{t+1} \approx x_t$ を維持しようとするあまり、プロンプト $c$ が指示する急激な状態変化 $c_{motion}$ を無視し、初期状態 $x_1$ に固着してしまう。我々はこの現象を "Static Death" と定義する。

\subsection{既存制御手法の限界}
この課題に対し、いくつかの既存アプローチが考えられるが、いずれも決定的な解決には至っていない。
\begin{itemize}
    \item \textbf{ControlNet等の空間制御}: 骨格情報や深度マップを入力として動作を強制する手法。強力であるが、適切な条件画像(Condition)を事前用意する必要があり、生成コストが高い。また、モデル自体の「動きにくさ」を解消するわけではないため、不自然な変形が生じやすい。
    \item \textbf{Prompt-to-Prompt等の注意制御}: Cross-Attentionマップを操作して対象物を置換・消去する手法。静止画では有効だが、動画においてはSelf-Attentionによる時間的参照が支配的であるため、効果が限定的である。
    \item \textbf{Global CFG Adjustment}: CFGスケールを一律に下げることでモデルの拘束を緩める手法。しかし、全体的にスケールを下げると画質や構造的一貫性が崩壊し、逆に上げると "Static Death" が強化されるというジレンマがある。
\end{itemize}

\section{提案手法: Frequency-Adaptive Hybrid Control}
前節の分析から、"Static Death" を克服するためには、(1) 生成プロセスの特定の段階(周波数帯域)においてのみ動的に介入し、(2) 時間的参照(記憶)を選択的に操作する手法が必要であることが示唆される。我々はこの方針に基づき、\textbf{Frequency-Adaptive Hybrid Control (FAHC)} を提案する。

\subsection{着想: 生成フェーズの分離}
拡散モデルの生成過程は、初期の低周波成分(大まかな構図・動き)の決定から、終盤の高周波成分(細部・質感)の決定へと進むことが知られている。
"Static Death" は、初期段階において「動き出そうとする力」が「留まろうとする慣性」に負けることで発生する。したがって、初期段階(低周波領域)においてのみ強力なエネルギー(Impulse)を与え、あるいは慣性を弱める(Relaxation)ことができれば、終盤の一貫性を損なうことなく、動的な遷移を実現できるはずである。

\subsection{Adaptive CFG Scaling: 動的エネルギー注入}
我々は、ガイダンススケール $w$ をノイズレベル $\sigma_t$ の関数として定義し、生成フェーズに応じた動的なエネルギー制御を行う。
標準的なCFGの式を再掲する。
\begin{equation}
\tilde{\epsilon}_\theta(x_t, c) = \epsilon_\theta(x_t, \varnothing) + w \cdot (\epsilon_\theta(x_t, c) - \epsilon_\theta(x_t, \varnothing))
\end{equation}
ここで、時間依存スケール $w(\sigma_t)$ を以下のように設計する。
\begin{equation}
w(\sigma_t) = w_{base} + \alpha \cdot \text{sgn}(\beta) |\beta| \cdot \sigma_t^p
\end{equation}

この数式の各項は物理的な意味を持つ。
\begin{itemize}
    \item \textbf{Impulse Mode ($\beta < 0$): 活性化エネルギーの供給} \\
    バク転のような高エネルギー動作には、静止摩擦係数を超えるような初期推力が必要である。$\beta < 0$ とすることで、生成初期($\sigma_t \approx 1$)に $w(\sigma_t)$ を一時的に増大(Boost)させる。これにより、プロンプト $c$ の強制力を極大化し、モデルの慣性を打ち破る「活性化エネルギー」を供給する。
    \item \textbf{Relaxation Mode ($\beta > 0$): 構造的可塑性の付与} \\
    逆に、消失タスクのように「存在」自体を消す場合、モデルの構造維持バイアスを弱める必要がある。$\beta > 0$ とすることで初期の $w(\sigma_t)$ を減少させ、モデルを「物理法則に従わない(消えてもよい)」状態にする。
    \item \textbf{Power Parameter ($p$): 緩和時間の制御} \\
    $p$ は介入の持続時間を制御する。$p=2.0$(急減衰)はインパルスを一瞬だけ与えて即座に安定させる「衝撃」として機能し、$p=0.7$(緩減衰)は動作の自由度を長く維持する「粘性低下」として機能する。本研究の実験により、この $p$ の値が生成品質(Robot Dancing vs Natural Motion)を分ける決定的な要因であることが判明した。
\end{itemize}

\subsection{Temporal Context Unlearning: 選択的忘却}
もう一つの障壁である「時間的参照(Self-Attention)」に対し、情報理論的な観点から介入を行う。
Attention機構は、過去のフレーム $K_{past}$ を参照することで情報をコピーするが、これは「過去の物体が存在し続ける」という強い情報を現在フレームに与え続けることと同義である。
消失タスクを成功させるには、この情報の信頼度を下げる(エントロピーを増大させる)必要がある。

そこで、Key/Valueキャッシュに対して時間軸方向のガウシアンブラーを適用する。
\begin{equation}
\tilde{K}_\tau = \sum_{i} K_{\tau+i} \cdot \mathcal{N}(i; 0, \sigma_{blur}^2)
\end{equation}
この操作は、信号処理の観点からはローパスフィルタリングであり、時間的に急峻な特徴量(特定のフレームにのみ存在する詳細情報)を平滑化する効果を持つ。結果として、モデルは過去の特定の瞬間の「鮮明な姿」を参照できなくなり、現在のプロンプト(「何もいない」)に従って、その場所を背景で埋める(インペインティングする)ことが容易になる。
我々はこれを \textbf{Temporal Context Unlearning} と呼び、明示的な学習を行わずにモデルの「記憶」を操作する手法として位置づける。

\section{実験・考察}
HunyuanVideoを用い、「消失(Disappearance)」と「バク転(Backflip)」の2つのタスクで手法の有効性を検証した。

\subsection{評価指標と目的}
本研究の主目的は、動画生成における「動的な変化(Dynamics)」と「静的な一貫性(Consitency)」という相反する要素を、導入した変数 $\beta$ (Intervention Strength) および $p$ (Decay Power) によって能動的に制御可能であることを実証することにある。
従来の固定パラメータでは到達不可能であった、プロンプトへの高い忠実度と映像の自然さを両立させる領域(Pareto Frontierの拡大)を、提案手法がカバーできることを定量的に示すため、以下の指標を用いた。
\begin{itemize}
    \item \textbf{Text-Video Alignment (CLIP Score)}: 
    生成された動画フレームとプロンプト(例: "A man performs a backflip")との意味的な類似度をCLIPモデルを用いて計算する。これはモデルが「どれだけプロンプトの指示(無理な動作)に従おうとしたか」を表す指標である。VideoMAE等の動作認識モデルは学習データ外の未知の動作に対して脆弱であるため、より汎用的なCLIPを用いた。
    \item \textbf{Visual Consistency (LPIPS)}: 
    隣接フレーム間の知覚的距離(LPIPS)を計算し、映像の時間的安定性を評価する。値が小さいほど滑らかで破綻が少ないことを意味するが、極端に小さい場合は "Static Death"(静止画化)を示唆する。
\end{itemize}

\subsection{定量的評価結果}
「バク転(Backflip)」タスクにおけるパラメータごとの定量評価結果を表\ref{tab:results}に示す。

\begin{table}[htbp]
  \centering
  \caption{バク転タスクにおける各パラメータ設定の定量評価比較。$\beta$ は介入強度、Blurは忘却強度を表す。VideoMAEスコアは "somersaulting" 等のターゲット動作クラスの確率を示す。}
  \label{tab:results}
  \begin{tabular}{lccccc}
    \hline
    Condition & $\beta$ & Blur & CLIP $\uparrow$ & LPIPS $\downarrow$ & VideoMAE $\uparrow$ \\
    \hline
    Baseline & 0.0 & 0.0 & 0.152 & \textbf{0.166} & 0.004 \\
    \textbf{Ours} & \textbf{0.9} & \textbf{0.6} & \textbf{0.187} & 0.244 & \textbf{0.221} \\
    High-Blur & 1.0 & 1.5 & 0.180 & 0.260 & 0.034 \\
    \hline
  \end{tabular}
\end{table}

表\ref{tab:results}より、ベースライン(Baseline)ではLPIPSが最も低く安定しているが、VideoMAEスコアはほぼゼロであり、アクション生成に失敗していることがわかる(Static Death)。
一方、提案手法(Ours)ではVideoMAEスコアがベースライン比で約55倍に向上しており、CLIP Scoreも最大値を示した。LPIPSは0.244と上昇しているが、これは静止状態からの脱却に伴う必然的な変化であり、映像崩壊(High-Blur条件の0.260以上)には至っていない。
High-Blur条件ではクリップスコアは高いものの、VideoMAEスコアが低下しており、過度な忘却が動作の一貫性を損なうことを示唆している。

以上の結果より、提案手法のパラメータ $\beta$ と $p$ を適切に設定することで、一貫性と動的変化のトレードオフを能動的に制御し、従来困難であったアクション生成が可能になることが確認された。
\end{document}
